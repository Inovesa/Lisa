%% Generated by Sphinx.
\def\sphinxdocclass{report}
\documentclass[letterpaper,10pt,openany,oneside,english]{sphinxmanual}
\ifdefined\pdfpxdimen
   \let\sphinxpxdimen\pdfpxdimen\else\newdimen\sphinxpxdimen
\fi \sphinxpxdimen=.75bp\relax

\usepackage[utf8]{inputenc}
\ifdefined\DeclareUnicodeCharacter
 \ifdefined\DeclareUnicodeCharacterAsOptional
  \DeclareUnicodeCharacter{"00A0}{\nobreakspace}
  \DeclareUnicodeCharacter{"2500}{\sphinxunichar{2500}}
  \DeclareUnicodeCharacter{"2502}{\sphinxunichar{2502}}
  \DeclareUnicodeCharacter{"2514}{\sphinxunichar{2514}}
  \DeclareUnicodeCharacter{"251C}{\sphinxunichar{251C}}
  \DeclareUnicodeCharacter{"2572}{\textbackslash}
 \else
  \DeclareUnicodeCharacter{00A0}{\nobreakspace}
  \DeclareUnicodeCharacter{2500}{\sphinxunichar{2500}}
  \DeclareUnicodeCharacter{2502}{\sphinxunichar{2502}}
  \DeclareUnicodeCharacter{2514}{\sphinxunichar{2514}}
  \DeclareUnicodeCharacter{251C}{\sphinxunichar{251C}}
  \DeclareUnicodeCharacter{2572}{\textbackslash}
 \fi
\fi
\usepackage{cmap}
\usepackage[T1]{fontenc}
\usepackage{amsmath,amssymb,amstext}
\usepackage{babel}
\usepackage{times}
\usepackage[Bjarne]{fncychap}
\usepackage[dontkeepoldnames]{sphinx}

\usepackage{geometry}

% Include hyperref last.
\usepackage{hyperref}
% Fix anchor placement for figures with captions.
\usepackage{hypcap}% it must be loaded after hyperref.
% Set up styles of URL: it should be placed after hyperref.
\urlstyle{same}

\addto\captionsenglish{\renewcommand{\figurename}{Fig.}}
\addto\captionsenglish{\renewcommand{\tablename}{Table}}
\addto\captionsenglish{\renewcommand{\literalblockname}{Listing}}

\addto\captionsenglish{\renewcommand{\literalblockcontinuedname}{continued from previous page}}
\addto\captionsenglish{\renewcommand{\literalblockcontinuesname}{continues on next page}}

\addto\extrasenglish{\def\pageautorefname{page}}

\setcounter{tocdepth}{1}



\title{Lisa Documentation}
\date{Jun 08, 2018}
\release{0.4.0.2}
\author{Patrick Schreiber}
\newcommand{\sphinxlogo}{\vbox{}}
\renewcommand{\releasename}{Release}
\makeindex

\begin{document}

\maketitle
\sphinxtableofcontents
\phantomsection\label{\detokenize{index::doc}}


Lisa is a Python module to easily plot and interact with Data from Inovesa result files (h5).


\chapter{Availability of Classes}
\label{\detokenize{availability:welcome-to-lisa-s-documentation}}\label{\detokenize{availability::doc}}\label{\detokenize{availability:availability-of-classes}}\begin{description}
\item[{The following Classes are available at top Level (using Lisa.\textless{}CLS\textgreater{}):}] \leavevmode\begin{itemize}
\item {} 
data.file.File

\item {} 
data.file.MultiFile

\item {} 
data.data.Data

\item {} 
plots.plots.SimplePlotter

\item {} 
plots.plots.MultiPlot

\item {} 
plots.plots.PhaseSpace

\item {} 
core.plots.plots.MultiPhaseSpaceMovie

\end{itemize}

\end{description}

Style is available through Lisa.plots.Style


\chapter{Simple Usage}
\label{\detokenize{simpleusage:simple-usage}}\label{\detokenize{simpleusage::doc}}
This is intended for someone who simply wants to plot something quickly for example.


\section{SimplePlotter}
\label{\detokenize{simpleusage:simpleplotter}}
Each Dataset in the Inovesa generated h5 files is a possible plot. Just call the corresponding function.

Plots for data with multiple axes expect a period as parameter. If this is omitted a meshplot is done.
The period is used to generate a slice in the time axis.

\begin{sphinxVerbatim}[commandchars=\\\{\}]
\PYG{k+kn}{import} \PYG{n+nn}{Lisa}
\PYG{n}{sp} \PYG{o}{=} \PYG{n}{Lisa}\PYG{o}{.}\PYG{n}{SimplePlotter}\PYG{p}{(}\PYG{l+s+s2}{\PYGZdq{}}\PYG{l+s+s2}{/path/to/h5}\PYG{l+s+s2}{\PYGZdq{}}\PYG{p}{)}
\PYG{n}{sp}\PYG{o}{.}\PYG{n}{bunch\PYGZus{}profile}\PYG{p}{(}\PYG{l+m+mi}{100}\PYG{p}{,} \PYG{n}{xunit}\PYG{o}{=}\PYG{l+s+s1}{\PYGZsq{}}\PYG{l+s+s1}{s}\PYG{l+s+s1}{\PYGZsq{}}\PYG{p}{)}
\PYG{n}{plt}\PYG{o}{.}\PYG{n}{show}\PYG{p}{(}\PYG{p}{)}
\PYG{n}{sp}\PYG{o}{.}\PYG{n}{energy\PYGZus{}spread}\PYG{p}{(}\PYG{p}{)}
\PYG{n}{plt}\PYG{o}{.}\PYG{n}{show}\PYG{p}{(}\PYG{p}{)}
\end{sphinxVerbatim}


\section{MultiPlot}
\label{\detokenize{simpleusage:multiplot}}
MultiPlot is the same as SimplePlotter except it works on multiple files.

\begin{sphinxVerbatim}[commandchars=\\\{\}]
\PYG{k+kn}{import} \PYG{n+nn}{Lisa}
\PYG{n}{mp} \PYG{o}{=} \PYG{n}{Lisa}\PYG{o}{.}\PYG{n}{MultiPlot}\PYG{p}{(}\PYG{p}{)}
\PYG{n}{mp}\PYG{o}{.}\PYG{n}{add\PYGZus{}file}\PYG{p}{(}\PYG{l+s+s2}{\PYGZdq{}}\PYG{l+s+s2}{/path/to/h5}\PYG{l+s+s2}{\PYGZdq{}}\PYG{p}{)}
\PYG{n}{mp}\PYG{o}{.}\PYG{n}{add\PYGZus{}file}\PYG{p}{(}\PYG{l+s+s2}{\PYGZdq{}}\PYG{l+s+s2}{/path/to/second/h5}\PYG{l+s+s2}{\PYGZdq{}}\PYG{p}{)}
\PYG{o}{.}\PYG{o}{.}\PYG{o}{.}
\PYG{n}{mp}\PYG{o}{.}\PYG{n}{energy\PYGZus{}spread}\PYG{p}{(}\PYG{p}{)}  \PYG{c+c1}{\PYGZsh{} or any other plot supported by SimplePlotter}
\end{sphinxVerbatim}


\section{File}
\label{\detokenize{simpleusage:file}}
File is an object encapsulating the h5 file.

It has a method for each of the DataGroups in the h5 file.

Each method takes a parameter specifying the dataset to use. This parameter has to be an attribute from
Lisa.Axis (e.g. Lisa.Axis.XAXIS for spaceaxis, Lisa.Axis.DATA for data etc.)

Inovesa Parameters are available via File.parameters. It will return an h5.Attribute instance if no
parameter is specified.

For recurring access to data it is a speed improvement to call File.preload\_full(“name\_of\_dataset”).
This will read all the data to memory for faster access.

If one does not specify an axis a DataContainer containing all the data there is in the requested DataGroup
will be returned. This object is iterable, subscriptable and has a get method that accepts Lisa.Axis properties.

Usage:

\begin{sphinxVerbatim}[commandchars=\\\{\}]
\PYG{k+kn}{import} \PYG{n+nn}{Lisa}
\PYG{n}{file} \PYG{o}{=} \PYG{n}{Lisa}\PYG{o}{.}\PYG{n}{File}\PYG{p}{(}\PYG{l+s+s2}{\PYGZdq{}}\PYG{l+s+s2}{/path/to/h5}\PYG{l+s+s2}{\PYGZdq{}}\PYG{p}{)}
\PYG{n}{bunch\PYGZus{}profile\PYGZus{}axis} \PYG{o}{=} \PYG{n}{file}\PYG{o}{.}\PYG{n}{bunch\PYGZus{}profile}\PYG{p}{(}\PYG{n}{Lisa}\PYG{o}{.}\PYG{n}{Axis}\PYG{o}{.}\PYG{n}{XAXIS}\PYG{p}{)}
\PYG{n}{bunch\PYGZus{}profile} \PYG{o}{=} \PYG{n}{file}\PYG{o}{.}\PYG{n}{bunch\PYGZus{}profile}\PYG{p}{(}\PYG{n}{Lisa}\PYG{o}{.}\PYG{n}{Axis}\PYG{o}{.}\PYG{n}{DATA}\PYG{p}{)}
\end{sphinxVerbatim}


\section{Data}
\label{\detokenize{simpleusage:data}}
Data is an object encapsulating a File object. The benefit of this is it converts data to the given unit.

\begin{sphinxVerbatim}[commandchars=\\\{\}]
\PYG{k+kn}{import} \PYG{n+nn}{Lisa}
\PYG{n}{data} \PYG{o}{=} \PYG{n}{Lisa}\PYG{o}{.}\PYG{n}{Data}\PYG{p}{(}\PYG{l+s+s2}{\PYGZdq{}}\PYG{l+s+s2}{/path/to/h5}\PYG{l+s+s2}{\PYGZdq{}}\PYG{p}{)}
\PYG{n}{data}\PYG{o}{.}\PYG{n}{bunch\PYGZus{}profile}\PYG{p}{(}\PYG{n}{Lisa}\PYG{o}{.}\PYG{n}{Axis}\PYG{o}{.}\PYG{n}{DATA}\PYG{p}{,} \PYG{n}{unit}\PYG{o}{=}\PYG{l+s+s2}{\PYGZdq{}}\PYG{l+s+s2}{c/s}\PYG{l+s+s2}{\PYGZdq{}}\PYG{p}{)}  \PYG{c+c1}{\PYGZsh{} c/s for coulomb per second.}
\PYG{n}{data}\PYG{o}{.}\PYG{n}{bunch\PYGZus{}position}\PYG{p}{(}\PYG{n}{Lisa}\PYG{o}{.}\PYG{n}{Axis}\PYG{o}{.}\PYG{n}{XAXIS}\PYG{p}{,} \PYG{n}{unit}\PYG{o}{=}\PYG{l+s+s2}{\PYGZdq{}}\PYG{l+s+s2}{m}\PYG{l+s+s2}{\PYGZdq{}}\PYG{p}{)}  \PYG{c+c1}{\PYGZsh{} m for meter}
\PYG{n}{data}\PYG{o}{.}\PYG{n}{bunch\PYGZus{}position}\PYG{p}{(}\PYG{n}{Lisa}\PYG{o}{.}\PYG{n}{Axis}\PYG{o}{.}\PYG{n}{XAXIS}\PYG{p}{,} \PYG{n}{unit}\PYG{o}{=}\PYG{l+s+s2}{\PYGZdq{}}\PYG{l+s+s2}{s}\PYG{l+s+s2}{\PYGZdq{}}\PYG{p}{)}  \PYG{c+c1}{\PYGZsh{} s for meter}
\end{sphinxVerbatim}

To get data as it is saved in the h5 file use unit=None.

It is possible to pass the unit as second parameter without a keyword. (For raw data (unit=None) this is not possible).


\section{PhaseSpace}
\label{\detokenize{simpleusage:phasespace}}
PhaseSpace is used to generate PhaseSpace plots or movies.

Use PhaseSpace.plot\_ps to plot a phasespace or use PhaseSpace.phase\_space\_movie to generate a PhaseSpace Movie.

It is also possible to generate a movie with subtracted mean phase space using PhaseSpace.microstructure\_movie.

Furthermore is it possible to extract “screenshots” from those movies by passing “extract\_slice” (see


\section{MultiPhaseSpaceMovie}
\label{\detokenize{simpleusage:multiphasespacemovie}}
This is used to generate PhaseSpace movies from phase spaces in multiple files.

The method to generate the movie is MultiPhaseSpaceMovie.create\_movie
\phantomsection\label{\detokenize{plots:module-plots}}\index{plots (module)}

\chapter{Inovesa Data Plots}
\label{\detokenize{plots:plots}}\label{\detokenize{plots::doc}}\label{\detokenize{plots:inovesa-data-plots}}\begin{quote}\begin{description}
\item[{Author}] \leavevmode
Patrick Schreiber

\end{description}\end{quote}
\index{warn() (in module plots)}

\begin{fulllineitems}
\phantomsection\label{\detokenize{plots:plots.warn}}\pysiglinewithargsret{\sphinxcode{plots.}\sphinxbfcode{warn}}{\emph{x}}{}
\end{fulllineitems}

\index{Deprecated (class in plots)}

\begin{fulllineitems}
\phantomsection\label{\detokenize{plots:plots.Deprecated}}\pysiglinewithargsret{\sphinxbfcode{class }\sphinxcode{plots.}\sphinxbfcode{Deprecated}}{\emph{Exception}}{}
\end{fulllineitems}



\begin{fulllineitems}
\pysigline{\sphinxbfcode{setup\_plots():}}
Setup Plots to look native in Latex documents. Subject to change.

\end{fulllineitems}

\index{SimplePlotter (class in plots)}

\begin{fulllineitems}
\phantomsection\label{\detokenize{plots:plots.SimplePlotter}}\pysiglinewithargsret{\sphinxbfcode{class }\sphinxcode{plots.}\sphinxbfcode{SimplePlotter}}{\emph{object}}{}~
\begin{DUlineblock}{0em}
\item[] A Simple Plot Helper Class.  
\item[] It takes a Filename to the Constructor.  
\item[] Each actual plotting function is decorated using plot or meshPlot. See those for additional  
\item[] parameters to each plot function.  
\end{DUlineblock}
\index{\_\_init\_\_() (plots.SimplePlotter method)}

\begin{fulllineitems}
\phantomsection\label{\detokenize{plots:plots.SimplePlotter.__init__}}\pysiglinewithargsret{\sphinxbfcode{\_\_init\_\_}}{\emph{self}, \emph{filef}, \emph{unit\_connector='in'}}{}
Initialise a Simple Plotter instance
\begin{quote}\begin{description}
\item[{Parameters}] \leavevmode\begin{itemize}
\item {} 
\sphinxstyleliteralstrong{filef} \textendash{} The file name of this Plotter or a Lisa.File instance

\item {} 
\sphinxstyleliteralstrong{unit\_connector} \textendash{} The string to use between label and unit.

\end{itemize}

\end{description}\end{quote}

\end{fulllineitems}

\index{plot() (plots.SimplePlotter method)}

\begin{fulllineitems}
\phantomsection\label{\detokenize{plots:plots.SimplePlotter.plot}}\pysiglinewithargsret{\sphinxbfcode{plot}}{\emph{func}}{}
Decorator to reuse plotting methods for different data. Calling one of the actual plot  
functions will result in calling this.  
This means the following options are available:

General Options (always use as keywords):
\begin{quote}\begin{description}
\item[{Parameters}] \leavevmode\begin{itemize}
\item {} 
\sphinxstyleliteralstrong{fig} \textendash{} (optional) the figure to plot in

\item {} 
\sphinxstyleliteralstrong{ax} \textendash{} (optional) the axis to use to plot

\item {} 
\sphinxstyleliteralstrong{label} \textendash{} (optional) the label for this plot (legend)

\item {} 
\sphinxstyleliteralstrong{scale\_factor} \textendash{} (optional) a scale factor Note: This does not modify the labels

\item {} 
\sphinxstyleliteralstrong{use\_offset} \textendash{} (optional) a bool if one wants an offset on yaxis or not

\item {} 
\sphinxstyleliteralstrong{force\_exponential\_x} \textendash{} (optional) a bool to force exponential notation on xaxis or not

\item {} 
\sphinxstyleliteralstrong{force\_exponential\_y} \textendash{} (optional) a bool to force exponential notation on yaxis or not

\item {} 
\sphinxstyleliteralstrong{fft} \textendash{} (optional) a bool to plot fft(data) instead of data or string for method to  
use in numpy.fft

\item {} 
\sphinxstyleliteralstrong{fft\_padding} \textendash{} (optional) an integer to specify how much 0 will be padded to the data  
before fft default fft

\item {} 
\sphinxstyleliteralstrong{abs} \textendash{} (optional) a boolean to select if plot absolute values or direct data

\item {} 
\sphinxstyleliteralstrong{plt\_args} \textendash{} (optional) dictionary with arguments to the displaying function

\item {} 
\sphinxstyleliteralstrong{x\_log} \textendash{} (optional) a boolean to set the x axis to log scale

\item {} 
\sphinxstyleliteralstrong{y\_log} \textendash{} (optional) a boolean to set the y axis to log scale

\item {} 
\sphinxstyleliteralstrong{idx\_range} \textendash{} (optional) a tuple with minimum and maximum index to plot

\end{itemize}

\end{description}\end{quote}

\end{fulllineitems}

\index{meshPlot() (plots.SimplePlotter method)}

\begin{fulllineitems}
\phantomsection\label{\detokenize{plots:plots.SimplePlotter.meshPlot}}\pysiglinewithargsret{\sphinxbfcode{meshPlot}}{\emph{func}}{}
Decorator to reuse plotting methods and to unify colormesh plots and normal line plots.  
Calling one of the actual plot functions will result in calling this.  
This means the following options are available:

General Options (always use as keywords):
\begin{quote}\begin{description}
\item[{Parameters}] \leavevmode\begin{itemize}
\item {} 
\sphinxstyleliteralstrong{fig} \textendash{} (optional) the figure to plot in

\item {} 
\sphinxstyleliteralstrong{ax} \textendash{} (optional) the axis to use to plot

\item {} 
\sphinxstyleliteralstrong{label} \textendash{} (optional) the label for this plot (legend) (if line plot)

\item {} 
\sphinxstyleliteralstrong{norm} \textendash{} (optional) mpl Norm object or one of {[}“linear”, “log”{]} to use for pcolormesh  
(default linear)

\item {} 
\sphinxstyleliteralstrong{colormap} \textendash{} (optional) the colormap for pcolormesh to use (default PuBu)

\item {} 
\sphinxstyleliteralstrong{force\_bad\_to\_min} \textendash{} (optional) force bad values (e.g. negative or zero in LogNorm) of  
colorbar to minimum color of colorbar

\item {} 
\sphinxstyleliteralstrong{force\_exponential\_x} \textendash{} (optional) a bool if one wants to force exponential notation on  
xaxis or not

\item {} 
\sphinxstyleliteralstrong{force\_exponential\_y} \textendash{} (optional) a bool if one wants to force exponential notation on  
yaxis or not

\item {} 
\sphinxstyleliteralstrong{plt\_args} \textendash{} (optional) dictionary with arguments to the displaying function

\item {} 
\sphinxstyleliteralstrong{period} \textendash{} (optional) the period to use. If not given will plot all data as pcolormesh  
(use parameters from plot)

\item {} 
\sphinxstyleliteralstrong{use\_index} \textendash{} (optional) Use period as index in data and not synchrotron period  
(default False)

\item {} 
\sphinxstyleliteralstrong{mean\_range} \textendash{} (optional) If given plot a normal plot but with data from mean of the  
given range (use parameters from plot)

\item {} 
\sphinxstyleliteralstrong{transpose} \textendash{} (optional) Transpose the 2d Plot (x-axis is time instead of y-axis)

\end{itemize}

\end{description}\end{quote}

\end{fulllineitems}

\index{energy\_spread() (plots.SimplePlotter method)}

\begin{fulllineitems}
\phantomsection\label{\detokenize{plots:plots.SimplePlotter.energy_spread}}\pysiglinewithargsret{\sphinxbfcode{energy\_spread}}{\emph{self}, \emph{**kwargs}}{}~\begin{description}
\item[{kwargs:  }] \leavevmode\begin{itemize}
\item {} 
xunit: possible values: “ts”, “seconds”, “raw”

\item {} 
yunit: possible values: “eV”, “raw”

\end{itemize}

\end{description}

\end{fulllineitems}

\index{bunch\_profile() (plots.SimplePlotter method)}

\begin{fulllineitems}
\phantomsection\label{\detokenize{plots:plots.SimplePlotter.bunch_profile}}\pysiglinewithargsret{\sphinxbfcode{bunch\_profile}}{\emph{self}, \emph{*args}, \emph{**kwargs}}{}~
\begin{DUlineblock}{0em}
\item[] Plot the bunch\_profile either as line plot or as pcolormesh  
\item[] to plot as line pass either the period as first argument or a keyword argument period  
\item[] Note: if yunit is passed period is in that unit, else in default value.  
\end{DUlineblock}
\begin{description}
\item[{kwargs: (first value is default)  }] \leavevmode\begin{itemize}
\item {} 
xunit: possible values: “meters”, “seconds”, “raw”

\item {} 
yunit: possible values: “ts”, “seconds”, “raw”

\item {} 
zunit: possible values: “coulomb”, “ampere”, “raw”

\item {} \begin{description}
\item[{pad\_zero: True or False. Pad data to zero to avoid white lines in plot (only considered  }] \leavevmode
if period is None or not given)

\end{description}

\end{itemize}

\end{description}

\end{fulllineitems}

\index{wake\_potential() (plots.SimplePlotter method)}

\begin{fulllineitems}
\phantomsection\label{\detokenize{plots:plots.SimplePlotter.wake_potential}}\pysiglinewithargsret{\sphinxbfcode{wake\_potential}}{\emph{self}, \emph{*args}, \emph{**kwargs}}{}~
\begin{DUlineblock}{0em}
\item[] Plot the wake\_potential either as line plot or as pcolormesh  
\item[] to plot as line pass either the period as first argument or a keyword argument period  
\item[] Note: if yunit is passed period is in that unit, else in default value.  
\end{DUlineblock}
\begin{description}
\item[{kwargs: (first value is default)  }] \leavevmode\begin{itemize}
\item {} 
xunit: possible values: “meters”, “seconds”, “raw”

\item {} 
yunit: possible values: “ts”, “seconds”, “raw”

\item {} 
zunit: possible values: “volt”, “raw”

\item {} \begin{description}
\item[{pad\_zero: True or False. Pad data to zero to avoid white lines in plot  }] \leavevmode
(only considered if period is None or not given)

\end{description}

\end{itemize}

\end{description}

\end{fulllineitems}

\index{bunch\_length() (plots.SimplePlotter method)}

\begin{fulllineitems}
\phantomsection\label{\detokenize{plots:plots.SimplePlotter.bunch_length}}\pysiglinewithargsret{\sphinxbfcode{bunch\_length}}{\emph{self}, \emph{**kwargs}}{}~\begin{description}
\item[{kwargs:  }] \leavevmode\begin{itemize}
\item {} 
xunit: possible values: “ts”, “seconds”, “raw”

\item {} 
yunit: possible values: “meters”, “secons”, “raw”

\end{itemize}

\end{description}

\end{fulllineitems}

\index{csr\_intensity() (plots.SimplePlotter method)}

\begin{fulllineitems}
\phantomsection\label{\detokenize{plots:plots.SimplePlotter.csr_intensity}}\pysiglinewithargsret{\sphinxbfcode{csr\_intensity}}{\emph{self}, \emph{**kwargs}}{}~\begin{description}
\item[{kwargs:  }] \leavevmode\begin{itemize}
\item {} 
xunit: possible values: “ts”, “seconds”, “raw”

\item {} 
yunit: possible values: “watt”, “raw”

\end{itemize}

\end{description}

\end{fulllineitems}

\index{bunch\_position() (plots.SimplePlotter method)}

\begin{fulllineitems}
\phantomsection\label{\detokenize{plots:plots.SimplePlotter.bunch_position}}\pysiglinewithargsret{\sphinxbfcode{bunch\_position}}{\emph{self}, \emph{**kwargs}}{}~\begin{description}
\item[{kwargs: (first value is default)  }] \leavevmode\begin{itemize}
\item {} 
xunit: possible values: “ts”, “seconds”, “raw”

\item {} 
yunit: possible values: “meters”, “seconds”, “raw”

\end{itemize}

\end{description}

\end{fulllineitems}

\index{bunch\_population() (plots.SimplePlotter method)}

\begin{fulllineitems}
\phantomsection\label{\detokenize{plots:plots.SimplePlotter.bunch_population}}\pysiglinewithargsret{\sphinxbfcode{bunch\_population}}{\emph{self}, \emph{**kwargs}}{}~\begin{description}
\item[{kwargs: (first value is default)  }] \leavevmode\begin{itemize}
\item {} 
xunit: possible values: “ts”, “seconds”, “raw”

\item {} 
yunit: possible values: “meters”, “seconds”, “raw”

\end{itemize}

\end{description}

\end{fulllineitems}

\index{csr\_spectrum() (plots.SimplePlotter method)}

\begin{fulllineitems}
\phantomsection\label{\detokenize{plots:plots.SimplePlotter.csr_spectrum}}\pysiglinewithargsret{\sphinxbfcode{csr\_spectrum}}{\emph{self}, \emph{*args}, \emph{**kwargs}}{}~
\begin{DUlineblock}{0em}
\item[] Plot the csr\_spectrum either as line plot or as pcolormesh  
\item[] to plot as line pass either the period as first argument or a keyword argument period  
\item[] Note: if yunit is passed period is in that unit, else in default value.  
\end{DUlineblock}
\begin{description}
\item[{kwargs: (first value is default)  }] \leavevmode\begin{itemize}
\item {} 
xunit: possible values: “ts”, “seconds”, “raw”

\item {} 
yunit: possible values: “hertz”, “raw”

\item {} 
zunit: possible values: “watt”, “raw”

\end{itemize}

\end{description}

\end{fulllineitems}

\index{energy\_profile() (plots.SimplePlotter method)}

\begin{fulllineitems}
\phantomsection\label{\detokenize{plots:plots.SimplePlotter.energy_profile}}\pysiglinewithargsret{\sphinxbfcode{energy\_profile}}{\emph{self}, \emph{*args}, \emph{**kwargs}}{}~
\begin{DUlineblock}{0em}
\item[] Plot the energy\_profile either as line plot or as pcolormesh  
\item[] to plot as line pass either the period as first argument or a keyword argument period  
\end{DUlineblock}
\begin{description}
\item[{kwargs: (first value is default)  }] \leavevmode\begin{itemize}
\item {} 
xunit: possible values: “eV”, “raw”

\item {} 
yunit: possible values: “ts”, “seconds”, “raw”

\item {} 
zunit: possible values: “coulomb”, “ampere”, “raw”

\end{itemize}

\end{description}

\end{fulllineitems}

\index{impedance() (plots.SimplePlotter method)}

\begin{fulllineitems}
\phantomsection\label{\detokenize{plots:plots.SimplePlotter.impedance}}\pysiglinewithargsret{\sphinxbfcode{impedance}}{\emph{self}, \emph{*args}, \emph{**kwargs}}{}
Plot Impedance (Fixed units). Real and Imaginary Part

\end{fulllineitems}

\index{video() (plots.SimplePlotter method)}

\begin{fulllineitems}
\phantomsection\label{\detokenize{plots:plots.SimplePlotter.video}}\pysiglinewithargsret{\sphinxbfcode{video}}{\emph{func}}{}
\end{fulllineitems}


\end{fulllineitems}

\index{MultiPlot (class in plots)}

\begin{fulllineitems}
\phantomsection\label{\detokenize{plots:plots.MultiPlot}}\pysiglinewithargsret{\sphinxbfcode{class }\sphinxcode{plots.}\sphinxbfcode{MultiPlot}}{\emph{object}}{}
Combine multiple files into one plot
\index{\_\_init\_\_() (plots.MultiPlot method)}

\begin{fulllineitems}
\phantomsection\label{\detokenize{plots:plots.MultiPlot.__init__}}\pysiglinewithargsret{\sphinxbfcode{\_\_init\_\_}}{\emph{self}}{}~
\begin{DUlineblock}{0em}
\item[] Creates MultiPlot Instance  
\item[] No Parameters (to add files use add\_file)  
\end{DUlineblock}

\end{fulllineitems}

\index{clone() (plots.MultiPlot method)}

\begin{fulllineitems}
\phantomsection\label{\detokenize{plots:plots.MultiPlot.clone}}\pysiglinewithargsret{\sphinxbfcode{clone}}{\emph{self}}{}
Return a copy of this instance

\end{fulllineitems}

\index{add\_file() (plots.MultiPlot method)}

\begin{fulllineitems}
\phantomsection\label{\detokenize{plots:plots.MultiPlot.add_file}}\pysiglinewithargsret{\sphinxbfcode{add\_file}}{\emph{self}, \emph{filename}, \emph{label=None}}{}
Add a file to this instance
\begin{quote}\begin{description}
\item[{Parameters}] \leavevmode\begin{itemize}
\item {} 
\sphinxstyleliteralstrong{filename} \textendash{} Path to the file

\item {} 
\sphinxstyleliteralstrong{label} \textendash{} (optional) the label for plots with this filename

\end{itemize}

\end{description}\end{quote}

\end{fulllineitems}

\index{reset() (plots.MultiPlot method)}

\begin{fulllineitems}
\phantomsection\label{\detokenize{plots:plots.MultiPlot.reset}}\pysiglinewithargsret{\sphinxbfcode{reset}}{\emph{self}}{}
Reset this instance

\end{fulllineitems}

\index{possible\_plots() (plots.MultiPlot method)}

\begin{fulllineitems}
\phantomsection\label{\detokenize{plots:plots.MultiPlot.possible_plots}}\pysiglinewithargsret{\sphinxbfcode{possible\_plots}}{\emph{self}}{}
List all possible plots

\end{fulllineitems}


\end{fulllineitems}

\index{PhaseSpace (class in plots)}

\begin{fulllineitems}
\phantomsection\label{\detokenize{plots:plots.PhaseSpace}}\pysiglinewithargsret{\sphinxbfcode{class }\sphinxcode{plots.}\sphinxbfcode{PhaseSpace}}{\emph{object}}{}
Plot PhaseSpaces of a single Inovesa result file
\index{\_\_init\_\_() (plots.PhaseSpace method)}

\begin{fulllineitems}
\phantomsection\label{\detokenize{plots:plots.PhaseSpace.__init__}}\pysiglinewithargsret{\sphinxbfcode{\_\_init\_\_}}{\emph{self}, \emph{file}}{}
\end{fulllineitems}

\index{eax() (plots.PhaseSpace method)}

\begin{fulllineitems}
\phantomsection\label{\detokenize{plots:plots.PhaseSpace.eax}}\pysiglinewithargsret{\sphinxbfcode{eax}}{\emph{self}}{}
\end{fulllineitems}

\index{xax() (plots.PhaseSpace method)}

\begin{fulllineitems}
\phantomsection\label{\detokenize{plots:plots.PhaseSpace.xax}}\pysiglinewithargsret{\sphinxbfcode{xax}}{\emph{self}}{}
\end{fulllineitems}

\index{ps\_data() (plots.PhaseSpace method)}

\begin{fulllineitems}
\phantomsection\label{\detokenize{plots:plots.PhaseSpace.ps_data}}\pysiglinewithargsret{\sphinxbfcode{ps\_data}}{\emph{self}, \emph{index}}{}
\end{fulllineitems}

\index{clone() (plots.PhaseSpace method)}

\begin{fulllineitems}
\phantomsection\label{\detokenize{plots:plots.PhaseSpace.clone}}\pysiglinewithargsret{\sphinxbfcode{clone}}{\emph{self}}{}~
\begin{DUlineblock}{0em}
\item[] Return a copy of this instance  
\item[] This effectively creates a new object with same file object  
\end{DUlineblock}

\end{fulllineitems}

\index{plot\_ps() (plots.PhaseSpace method)}

\begin{fulllineitems}
\phantomsection\label{\detokenize{plots:plots.PhaseSpace.plot_ps}}\pysiglinewithargsret{\sphinxbfcode{plot\_ps}}{\emph{self}, \emph{index}}{}
Plot the phasespace.
\begin{quote}\begin{description}
\item[{Parameters}] \leavevmode
\sphinxstyleliteralstrong{index} \textendash{} the index of the dataset (in timeaxis)

\end{description}\end{quote}

\end{fulllineitems}

\index{center\_of\_mass() (plots.PhaseSpace method)}

\begin{fulllineitems}
\phantomsection\label{\detokenize{plots:plots.PhaseSpace.center_of_mass}}\pysiglinewithargsret{\sphinxbfcode{center\_of\_mass}}{\emph{self}, \emph{xax}, \emph{yax}}{}
\end{fulllineitems}

\index{phase\_space\_movie() (plots.PhaseSpace method)}

\begin{fulllineitems}
\phantomsection\label{\detokenize{plots:plots.PhaseSpace.phase_space_movie}}\pysiglinewithargsret{\sphinxbfcode{phase\_space\_movie}}{\emph{self}, \emph{path=None}, \emph{fr\_idx=None}, \emph{to\_idx=None}, \emph{fps=20}, \emph{plot\_area\_width=None dpi=200}, \emph{csr\_intensity=False}, \emph{bunch\_profile=False}, \emph{cmap="inferno"}, \emph{clim=None}, \emph{extract\_slice=None}, \emph{**kwargs}}{}
Plot a movie of the evolving phasespace
\begin{quote}\begin{description}
\item[{Parameters}] \leavevmode\begin{itemize}
\item {} 
\sphinxstyleliteralstrong{path} \textendash{} Path to a movie file to save to if None: do not save, just return the  
animation object

\item {} 
\sphinxstyleliteralstrong{fr\_idx} \textendash{} Index in the phasespace data to start video

\item {} 
\sphinxstyleliteralstrong{to\_idx} \textendash{} Index in the phasespace data to stop video

\item {} 
\sphinxstyleliteralstrong{fps} \textendash{} Frames per second of the output video

\item {} 
\sphinxstyleliteralstrong{plot\_area\_width} \textendash{} The width in pixel to plot around com. If None will plot full  
phasespace. (Will plot from com-plot\_area\_width/2 to com+plot\_area\_width/2  
in space and energy)

\item {} 
\sphinxstyleliteralstrong{dpi} \textendash{} Dots per inch of output video

\item {} 
\sphinxstyleliteralstrong{csr\_intensity} \textendash{} Also plot CSR intensity and marker of current position

\item {} 
\sphinxstyleliteralstrong{bunch\_profile} \textendash{} Also plot the bunch profile for current synchrotron period

\item {} 
\sphinxstyleliteralstrong{cmap} \textendash{} Colormap to use

\item {} 
\sphinxstyleliteralstrong{clim} \textendash{} Maximum in fraction of global min/max to use as colormap limits

\item {} 
\sphinxstyleliteralstrong{extract\_slice} \textendash{} Extract a slice and do no movie, just return the plot. This is a  
string in format idx:int for an actual slice or ts:float for a specific synchrotron  
period (or the nearest value). Or it is an integer for a slice index  
(same as idx:int)

\item {} 
\sphinxstyleliteralstrong{**kwargs} \textendash{} 
Keyword arguments passed to create\_animation


\end{itemize}

\item[{Returns}] \leavevmode
animation object

\end{description}\end{quote}

\end{fulllineitems}

\index{microstructure\_movie() (plots.PhaseSpace method)}

\begin{fulllineitems}
\phantomsection\label{\detokenize{plots:plots.PhaseSpace.microstructure_movie}}\pysiglinewithargsret{\sphinxbfcode{microstructure\_movie}}{\emph{self}, \emph{path=None}, \emph{fr\_idx=None}, \emph{to\_idx=None}, \emph{mean\_range=(None}, \emph{None) fps=20}, \emph{plot\_area\_width=None}, \emph{dpi=200}, \emph{csr\_intensity=False}, \emph{bunch\_profile=False}, \emph{cmap="RdBu\_r"}, \emph{clim=None}, \emph{extract\_slice=None}, \emph{**kwargs}}{}
Plot the difference between the mean phasespace and the current snapshot as video.
\begin{quote}\begin{description}
\item[{Parameters}] \leavevmode\begin{itemize}
\item {} 
\sphinxstyleliteralstrong{path} \textendash{} Path to a movie file to save to if None: do not save, just return the animation  
object

\item {} 
\sphinxstyleliteralstrong{fr\_idx} \textendash{} Index in the phasespace data to start video

\item {} 
\sphinxstyleliteralstrong{to\_idx} \textendash{} Index in the phasespace data to stop video

\item {} 
\sphinxstyleliteralstrong{mean\_range} \textendash{} The min index and max index to use when calculating the mean of the  
phasespace

\item {} 
\sphinxstyleliteralstrong{fps} \textendash{} Frames per second of the output video

\item {} 
\sphinxstyleliteralstrong{plot\_area\_width} \textendash{} The width in pixel to plot around com. If None will plot full  
phasespace.  
(Will plot from com-plot\_area\_width/2 to com+plot\_area\_width/2 in space and energy)

\item {} 
\sphinxstyleliteralstrong{dpi} \textendash{} Dots per inch of output video

\item {} 
\sphinxstyleliteralstrong{csr\_intensity} \textendash{} Also plot CSR intensity and marker of current position

\item {} 
\sphinxstyleliteralstrong{bunch\_profile} \textendash{} Also plot the bunch profile for current synchrotron period

\item {} 
\sphinxstyleliteralstrong{cmap} \textendash{} Colormap to use

\item {} 
\sphinxstyleliteralstrong{clim} \textendash{} Maximum in fraction of global min/max to use as colormap limits

\item {} 
\sphinxstyleliteralstrong{extract\_slice} \textendash{} Extract a slice and do no movie, just return the plot. This is a  
string in format idx:int for an actual slice or ts:float for a specific synchrotron  
period (or the nearest value). Or it is an integer for a slice index  
(same as idx:int)

\item {} 
\sphinxstyleliteralstrong{**kwargs} \textendash{} 
Keyword arguments passed to create\_animation


\end{itemize}

\item[{Returns}] \leavevmode
animation object

\end{description}\end{quote}

\end{fulllineitems}


\end{fulllineitems}

\index{MultiPhaseSpaceMovie (class in plots)}

\begin{fulllineitems}
\phantomsection\label{\detokenize{plots:plots.MultiPhaseSpaceMovie}}\pysiglinewithargsret{\sphinxbfcode{class }\sphinxcode{plots.}\sphinxbfcode{MultiPhaseSpaceMovie}}{\emph{object}}{}~
\begin{DUlineblock}{0em}
\item[] Create Phasespace of multiple Files  
\item[] Useful to check the phasespace over multiple currenst (spectrogram)  
\end{DUlineblock}
\index{\_\_init\_\_() (plots.MultiPhaseSpaceMovie method)}

\begin{fulllineitems}
\phantomsection\label{\detokenize{plots:plots.MultiPhaseSpaceMovie.__init__}}\pysiglinewithargsret{\sphinxbfcode{\_\_init\_\_}}{\emph{self}, \emph{path}}{}
\end{fulllineitems}

\index{create\_movie() (plots.MultiPhaseSpaceMovie method)}

\begin{fulllineitems}
\phantomsection\label{\detokenize{plots:plots.MultiPhaseSpaceMovie.create_movie}}\pysiglinewithargsret{\sphinxbfcode{create\_movie}}{\emph{self}, \emph{filename}, \emph{dpi=200}, \emph{size\_inches=(5.5}, \emph{5.5)}, \emph{fps=30}, \emph{autorescale=False}}{}
Create a Movie of phasespaces
\begin{quote}\begin{description}
\item[{Parameters}] \leavevmode\begin{itemize}
\item {} 
\sphinxstyleliteralstrong{filename} \textendash{} The filename to use for the produces video file  
(if None, a moviepy video object will be returned)

\item {} 
\sphinxstyleliteralstrong{dpi}\sphinxstyleliteralstrong{(}\sphinxstyleliteralstrong{=200}\sphinxstyleliteralstrong{)} \textendash{} the dpi of the produces video

\item {} 
\sphinxstyleliteralstrong{5.5}\sphinxstyleliteralstrong{)} (\sphinxstyleliteralemphasis{size\_inches}\sphinxstyleliteralemphasis{(}\sphinxstyleliteralemphasis{=5.5}\sphinxstyleliteralemphasis{,}) \textendash{} tuple used for size in inces of the video

\item {} 
\sphinxstyleliteralstrong{fps}\sphinxstyleliteralstrong{(}\sphinxstyleliteralstrong{=30}\sphinxstyleliteralstrong{)} \textendash{} the frames per second to use

\item {} 
\sphinxstyleliteralstrong{autorescale}\sphinxstyleliteralstrong{(}\sphinxstyleliteralstrong{=False}\sphinxstyleliteralstrong{)} \textendash{} if True will autorescale each frame (will not make sense if  
you want to compare)

\end{itemize}

\item[{Returns}] \leavevmode
True if file produced, moviepy video instance if None as filename was given

\end{description}\end{quote}

\end{fulllineitems}


\end{fulllineitems}

\phantomsection\label{\detokenize{file:module-file}}\index{file (module)}

\chapter{Inovesa Result Files}
\label{\detokenize{file:file}}\label{\detokenize{file:inovesa-result-files}}\label{\detokenize{file::doc}}\begin{quote}\begin{description}
\item[{Author}] \leavevmode
Patrick Schreiber

\end{description}\end{quote}
\index{AttributedNPArray (class in file)}

\begin{fulllineitems}
\phantomsection\label{\detokenize{file:file.AttributedNPArray}}\pysiglinewithargsret{\sphinxbfcode{class }\sphinxcode{file.}\sphinxbfcode{AttributedNPArray}}{\emph{np.ndarray}}{}
Simple Wrarpper around numpy.ndarray to include h5 attrs attribute

\end{fulllineitems}

\index{registered() (in module file)}

\begin{fulllineitems}
\phantomsection\label{\detokenize{file:file.registered}}\pysiglinewithargsret{\sphinxcode{file.}\sphinxbfcode{registered}}{\emph{cls}}{}
Registeres properties of one class into another class.registered\_properties

\end{fulllineitems}

\index{AxisSelector (class in file)}

\begin{fulllineitems}
\phantomsection\label{\detokenize{file:file.AxisSelector}}\pysiglinewithargsret{\sphinxbfcode{class }\sphinxcode{file.}\sphinxbfcode{AxisSelector}}{\emph{object}}{}~\index{\_\_init\_\_() (file.AxisSelector method)}

\begin{fulllineitems}
\phantomsection\label{\detokenize{file:file.AxisSelector.__init__}}\pysiglinewithargsret{\sphinxbfcode{\_\_init\_\_}}{\emph{self}, \emph{version}}{}
\end{fulllineitems}

\index{all\_for() (file.AxisSelector method)}

\begin{fulllineitems}
\phantomsection\label{\detokenize{file:file.AxisSelector.all_for}}\pysiglinewithargsret{\sphinxbfcode{all\_for}}{\emph{self}, \emph{group}}{}
\end{fulllineitems}


\end{fulllineitems}

\index{DataError (class in file)}

\begin{fulllineitems}
\phantomsection\label{\detokenize{file:file.DataError}}\pysiglinewithargsret{\sphinxbfcode{class }\sphinxcode{file.}\sphinxbfcode{DataError}}{\emph{Exception}}{}
\end{fulllineitems}

\index{DataContainer (class in file)}

\begin{fulllineitems}
\phantomsection\label{\detokenize{file:file.DataContainer}}\pysiglinewithargsret{\sphinxbfcode{class }\sphinxcode{file.}\sphinxbfcode{DataContainer}}{\emph{object}}{}~\index{\_\_init\_\_() (file.DataContainer method)}

\begin{fulllineitems}
\phantomsection\label{\detokenize{file:file.DataContainer.__init__}}\pysiglinewithargsret{\sphinxbfcode{\_\_init\_\_}}{\emph{self}, \emph{data\_dict}, \emph{list\_of\_elements}}{}
\end{fulllineitems}

\index{get() (file.DataContainer method)}

\begin{fulllineitems}
\phantomsection\label{\detokenize{file:file.DataContainer.get}}\pysiglinewithargsret{\sphinxbfcode{get}}{\emph{self}, \emph{item}, \emph{default}}{}
\end{fulllineitems}


\end{fulllineitems}

\index{File (class in file)}

\begin{fulllineitems}
\phantomsection\label{\detokenize{file:file.File}}\pysiglinewithargsret{\sphinxbfcode{class }\sphinxcode{file.}\sphinxbfcode{File}}{\emph{object}}{}
Wrapper around a h5 File created by Inovesa

Each Datagroup is a method of this object. To get a special axis of each group  
use additional parameters.

If one parameter is supplied the data will be returned as HDF5.Dataset object or, if preloaded,  
as AttributedNPArray.

If no or more than one parameter is supplied the data will be returned as DataContainer  
object containing the HDF5.Dataset or AttributedNPArray objects.

This is not completely true for File.parameters.

\begin{sphinxVerbatim}[commandchars=\\\{\}]
\PYG{n}{f} \PYG{o}{=} \PYG{n}{File}\PYG{p}{(}\PYG{l+s+s2}{\PYGZdq{}}\PYG{l+s+s2}{path/to/file}\PYG{l+s+s2}{\PYGZdq{}}\PYG{p}{)}  
\PYG{n}{tax} \PYG{o}{=} \PYG{n}{f}\PYG{o}{.}\PYG{n}{bunch\PYGZus{}profile}\PYG{p}{(}\PYG{n}{Axis}\PYG{o}{.}\PYG{n}{TIME}\PYG{p}{)}  
\PYG{n}{xax} \PYG{o}{=} \PYG{n}{f}\PYG{o}{.}\PYG{n}{bunch\PYGZus{}profile}\PYG{p}{(}\PYG{n}{Axis}\PYG{o}{.}\PYG{n}{XAXIS}\PYG{p}{)}  
\PYG{n}{data} \PYG{o}{=} \PYG{n}{f}\PYG{o}{.}\PYG{n}{bunch\PYGZus{}profile}\PYG{p}{(}\PYG{n}{Axis}\PYG{o}{.}\PYG{n}{DATA}\PYG{p}{)}  
\PYG{n+nb}{all} \PYG{o}{=} \PYG{n}{f}\PYG{o}{.}\PYG{n}{bunch\PYGZus{}profile}\PYG{p}{(}\PYG{p}{)}  
\PYG{n}{axes} \PYG{o}{=} \PYG{n}{f}\PYG{o}{.}\PYG{n}{bunch\PYGZus{}profile}\PYG{p}{(}\PYG{n}{Axis}\PYG{o}{.}\PYG{n}{TIME}\PYG{p}{,} \PYG{n}{Axis}\PYG{o}{.}\PYG{n}{XAXIS}\PYG{p}{)}  
\end{sphinxVerbatim}

Method names are mostly the datagroup-names with underscores instead of camelcase  
(e.g. bunch\_profile for BunchProfile)

Available Groups:
\begin{itemize}
\item {} 
energy\_spread

\item {} 
bunch\_length

\item {} 
bunch\_position

\item {} 
bunch\_population

\item {} 
bunch\_profile

\item {} 
csr\_intensity

\item {} 
csr\_spectrum

\item {} 
energy\_profile

\item {} 
impedance

\item {} 
particles

\item {} 
phase\_space

\item {} 
wake\_potential

\item {} 
source\_map (only Inovesa version 0.15 and above)

\item {} \begin{itemize}
\item {} 
parameters

\end{itemize}

\end{itemize}
\begin{itemize}
\item {} 
parameters does not expose the exact same interface. Use File.parameters() to get the

\end{itemize}

HDF5 Attribute Object. Use File.parameters(param1, param2 …) to get either a single parameter  
(with the datatype that this parameter is saved in the hdf5 file) or a DataContainer object  
depending on how much params were passed.
\index{\_\_init\_\_() (file.File method)}

\begin{fulllineitems}
\phantomsection\label{\detokenize{file:file.File.__init__}}\pysiglinewithargsret{\sphinxbfcode{\_\_init\_\_}}{\emph{self}, \emph{filename}}{}
Create File object
\begin{quote}\begin{description}
\item[{Parameters}] \leavevmode
\sphinxstyleliteralstrong{filename} \textendash{} The filename of the Inovesa result file

\end{description}\end{quote}

\end{fulllineitems}

\index{preload\_full() (file.File method)}

\begin{fulllineitems}
\phantomsection\label{\detokenize{file:file.File.preload_full}}\pysiglinewithargsret{\sphinxbfcode{preload\_full}}{\emph{self}, \emph{what}, \emph{axis=None}}{}
Preload Data into memory. This will speed up recurring reads by a lot

\end{fulllineitems}


\end{fulllineitems}

\index{MultiFile (class in file)}

\begin{fulllineitems}
\phantomsection\label{\detokenize{file:file.MultiFile}}\pysiglinewithargsret{\sphinxbfcode{class }\sphinxcode{file.}\sphinxbfcode{MultiFile}}{\emph{object}}{}
Multiple File container for whole directories
\index{\_\_init\_\_() (file.MultiFile method)}

\begin{fulllineitems}
\phantomsection\label{\detokenize{file:file.MultiFile.__init__}}\pysiglinewithargsret{\sphinxbfcode{\_\_init\_\_}}{\emph{self}, \emph{path}, \emph{pattern=None}, \emph{sorter=None}}{}~\begin{quote}\begin{description}
\item[{Parameters}] \leavevmode\begin{itemize}
\item {} 
\sphinxstyleliteralstrong{path} \textendash{} The path to search in

\item {} 
\sphinxstyleliteralstrong{pattern} \textendash{} The pattern to search for (has to be accepted by glob) (None for no pattern)

\item {} 
\sphinxstyleliteralstrong{sorter} \textendash{} The sorting method to use. (None for default)

\end{itemize}

\end{description}\end{quote}

\end{fulllineitems}

\index{set\_sorter() (file.MultiFile method)}

\begin{fulllineitems}
\phantomsection\label{\detokenize{file:file.MultiFile.set_sorter}}\pysiglinewithargsret{\sphinxbfcode{set\_sorter}}{\emph{self}, \emph{sorter}}{}
Set the sorting method
\begin{quote}\begin{description}
\item[{Parameters}] \leavevmode
\sphinxstyleliteralstrong{sorter} \textendash{} A callable that accepts a list of files as strings and returns a sorted list

\end{description}\end{quote}

\end{fulllineitems}

\index{strlst() (file.MultiFile method)}

\begin{fulllineitems}
\phantomsection\label{\detokenize{file:file.MultiFile.strlst}}\pysiglinewithargsret{\sphinxbfcode{strlst}}{\emph{self}, \emph{sorted=True}}{}
Get File list as string list
\begin{quote}\begin{description}
\item[{Parameters}] \leavevmode
\sphinxstyleliteralstrong{sorted} \textendash{} True to sort the list

\end{description}\end{quote}

\end{fulllineitems}

\index{objlst() (file.MultiFile method)}

\begin{fulllineitems}
\phantomsection\label{\detokenize{file:file.MultiFile.objlst}}\pysiglinewithargsret{\sphinxbfcode{objlst}}{\emph{self}, \emph{sorted=True}}{}
Get File list as Lisa.File object list
\begin{quote}\begin{description}
\item[{Parameters}] \leavevmode
\sphinxstyleliteralstrong{sorted} \textendash{} True to sort the list

\end{description}\end{quote}

\end{fulllineitems}


\end{fulllineitems}

\phantomsection\label{\detokenize{data:module-data}}\index{data (module)}

\chapter{Data with Units}
\label{\detokenize{data:data-with-units}}\label{\detokenize{data::doc}}\label{\detokenize{data:data}}\begin{quote}\begin{description}
\item[{Author}] \leavevmode
Patrick Schreiber

\end{description}\end{quote}
\index{Data (class in data)}

\begin{fulllineitems}
\phantomsection\label{\detokenize{data:data.Data}}\pysiglinewithargsret{\sphinxbfcode{class }\sphinxcode{data.}\sphinxbfcode{Data}}{\emph{object}}{}
Get Data from Inovesa Result Datafiles with specified unit.

Use the same attributes that are available for Lisa.File objects.

Usage:

\begin{sphinxVerbatim}[commandchars=\\\{\}]
\PYG{n}{d} \PYG{o}{=} \PYG{n}{Lisa}\PYG{o}{.}\PYG{n}{Data}\PYG{p}{(}\PYG{l+s+s2}{\PYGZdq{}}\PYG{l+s+s2}{/path/to/file}\PYG{l+s+s2}{\PYGZdq{}}\PYG{p}{)}  
\PYG{n}{d}\PYG{o}{.}\PYG{n}{bunch\PYGZus{}profile}\PYG{p}{(}\PYG{n}{Lisa}\PYG{o}{.}\PYG{n}{Axis}\PYG{o}{.}\PYG{n}{DATA}\PYG{p}{,} \PYG{n}{unit}\PYG{o}{=}\PYG{l+s+s2}{\PYGZdq{}}\PYG{l+s+s2}{c}\PYG{l+s+s2}{\PYGZdq{}}\PYG{p}{)}  \PYG{c+c1}{\PYGZsh{} unit=\PYGZdq{}c\PYGZdq{} for coulomb  }
\PYG{n}{d}\PYG{o}{.}\PYG{n}{bunch\PYGZus{}profile}\PYG{p}{(}\PYG{n}{Lisa}\PYG{o}{.}\PYG{n}{Axis}\PYG{o}{.}\PYG{n}{XAXIS}\PYG{p}{,} \PYG{n}{unit}\PYG{o}{=}\PYG{l+s+s2}{\PYGZdq{}}\PYG{l+s+s2}{s}\PYG{l+s+s2}{\PYGZdq{}}\PYG{p}{)}  \PYG{c+c1}{\PYGZsh{} unit=\PYGZdq{}s\PYGZdq{} for second XAXIS for space axis  }
\end{sphinxVerbatim}


\sphinxstrong{See also:}


Lisa.File


\index{\_\_init\_\_() (data.Data method)}

\begin{fulllineitems}
\phantomsection\label{\detokenize{data:data.Data.__init__}}\pysiglinewithargsret{\sphinxbfcode{\_\_init\_\_}}{\emph{self}, \emph{p\_file}}{}~\begin{quote}\begin{description}
\item[{Parameters}] \leavevmode
\sphinxstyleliteralstrong{p\_file} \textendash{} either a filename of a Lisa.File object

\end{description}\end{quote}

\end{fulllineitems}

\index{\_\_getattr\_\_() (data.Data method)}

\begin{fulllineitems}
\phantomsection\label{\detokenize{data:data.Data.__getattr__}}\pysiglinewithargsret{\sphinxbfcode{\_\_getattr\_\_}}{\emph{self}, \emph{attr}}{}
Convert to correct unit.
\begin{quote}\begin{description}
\item[{Parameters}] \leavevmode\begin{itemize}
\item {} 
\sphinxstyleliteralstrong{idx} \textendash{} An axis specification from Lisa.Axis

\item {} 
\sphinxstyleliteralstrong{unit} (\sphinxstyleliteralemphasis{string}) \textendash{} Use this as second argument or kwarg

\item {} 
\sphinxstyleliteralstrong{sub\_idx} \textendash{} (kwarg) the index in the h5object (if File returns {[}Dataset1, Dataset2{]}

\end{itemize}

\end{description}\end{quote}

then idx=0 and sub\_idx is given will result in Dataset1{[}sub\_idx{]})  
This will speed up if only part of the data is used.

\end{fulllineitems}

\index{unit\_factor() (data.Data method)}

\begin{fulllineitems}
\phantomsection\label{\detokenize{data:data.Data.unit_factor}}\pysiglinewithargsret{\sphinxbfcode{unit\_factor}}{\emph{self}, \emph{data}, \emph{axis}, \emph{unit}}{}
Get the factor to calculate values in the correct physical unit
\begin{quote}\begin{description}
\item[{Parameters}] \leavevmode\begin{itemize}
\item {} 
\sphinxstyleliteralstrong{data} \textendash{} The data to get the factor for (e.g. bunch\_profile)

\item {} 
\sphinxstyleliteralstrong{axis} \textendash{} The Axis object to select what axis to get the factor for

\item {} 
\sphinxstyleliteralstrong{unit} \textendash{} The actual unit

\end{itemize}

\item[{Returns}] \leavevmode
Factor to get the data in physical units (physical=data*factor)

\end{description}\end{quote}

\end{fulllineitems}


\end{fulllineitems}

\phantomsection\label{\detokenize{config:module-config}}\index{config (module)}

\chapter{Plot Configuration}
\label{\detokenize{config:config}}\label{\detokenize{config:plot-configuration}}\label{\detokenize{config::doc}}\begin{quote}\begin{description}
\item[{Author}] \leavevmode
Patrick Schreiber

\end{description}\end{quote}
\index{Alias (class in config)}

\begin{fulllineitems}
\phantomsection\label{\detokenize{config:config.Alias}}\pysigline{\sphinxbfcode{class }\sphinxcode{config.}\sphinxbfcode{Alias}}
Alias to another method

\end{fulllineitems}

\index{update\_line\_color() (in module config)}

\begin{fulllineitems}
\phantomsection\label{\detokenize{config:config.update_line_color}}\pysiglinewithargsret{\sphinxcode{config.}\sphinxbfcode{update\_line\_color}}{\emph{ax}, \emph{w}, \emph{v}}{}
\end{fulllineitems}

\index{StyleError (class in config)}

\begin{fulllineitems}
\phantomsection\label{\detokenize{config:config.StyleError}}\pysiglinewithargsret{\sphinxbfcode{class }\sphinxcode{config.}\sphinxbfcode{StyleError}}{\emph{Exception}}{}
\end{fulllineitems}

\index{Palette (class in config)}

\begin{fulllineitems}
\phantomsection\label{\detokenize{config:config.Palette}}\pysiglinewithargsret{\sphinxbfcode{class }\sphinxcode{config.}\sphinxbfcode{Palette}}{\emph{object}}{}~\index{next() (config.Palette method)}

\begin{fulllineitems}
\phantomsection\label{\detokenize{config:config.Palette.next}}\pysiglinewithargsret{\sphinxbfcode{next}}{\emph{self}}{}
\end{fulllineitems}

\index{reset() (config.Palette method)}

\begin{fulllineitems}
\phantomsection\label{\detokenize{config:config.Palette.reset}}\pysiglinewithargsret{\sphinxbfcode{reset}}{\emph{self}}{}
\end{fulllineitems}


\end{fulllineitems}

\index{Style (class in config)}

\begin{fulllineitems}
\phantomsection\label{\detokenize{config:config.Style}}\pysiglinewithargsret{\sphinxbfcode{class }\sphinxcode{config.}\sphinxbfcode{Style}}{\emph{dict}}{}
Style for Matplotlib axes.  
It is essentially a dictionary with overriden methods.

Usage (assume ax is an matplotlib axes object):

\begin{sphinxVerbatim}[commandchars=\\\{\}]
\PYG{n}{style} \PYG{o}{=} \PYG{n}{Style}\PYG{p}{(}\PYG{p}{)}  
\PYG{n}{style}\PYG{o}{.}\PYG{n}{apply\PYGZus{}to\PYGZus{}ax}\PYG{p}{(}\PYG{n}{ax}\PYG{p}{)}  
\PYG{n}{style}\PYG{o}{.}\PYG{n}{update}\PYG{p}{(}\PYG{p}{\PYGZob{}}\PYG{n}{selector}\PYG{p}{:} \PYG{n}{value}\PYG{p}{\PYGZcb{}}\PYG{p}{)}  \PYG{c+c1}{\PYGZsh{} or use style.update\PYGZus{}style  }
\PYG{n}{style}\PYG{p}{[}\PYG{l+s+s1}{\PYGZsq{}}\PYG{l+s+s1}{selector}\PYG{l+s+s1}{\PYGZsq{}}\PYG{p}{]}\PYG{o}{=}\PYG{n}{value}  \PYG{c+c1}{\PYGZsh{} this also updates the axes object  }
\PYG{c+c1}{\PYGZsh{} if you have done some modifications to the plot yourself and want to update the style  }
\PYG{n}{again} \PYG{n}{use} \PYG{n}{style}\PYG{o}{.}\PYG{n}{reapply}\PYG{p}{(}\PYG{p}{)}  
\end{sphinxVerbatim}

\begin{sphinxadmonition}{note}{Note:}
instead of working with the Style object directly you can use ax.current\_style after  
you applied a style
\end{sphinxadmonition}

\begin{sphinxadmonition}{note}{Note:}
You can apply one Style object to more than one axes object. Just call apply\_to\_ax  
multiple times or with a list of axes. If you applied one Style to multiple axes objects  
and you use axes.current\_style instead of the style object directly the updates/changes  
will nonetheless be applied to all applied axes objects.
\end{sphinxadmonition}
\index{apply\_to\_ax() (config.Style method)}

\begin{fulllineitems}
\phantomsection\label{\detokenize{config:config.Style.apply_to_ax}}\pysiglinewithargsret{\sphinxbfcode{apply\_to\_ax}}{\emph{self}, \emph{ax}}{}~
\begin{DUlineblock}{0em}
\item[] Apply this style to the given axes object  
\item[] You can apply to multiple axes objects by calling this multiple times or with a list of  
\end{DUlineblock}

objects
\begin{quote}\begin{description}
\item[{Parameters}] \leavevmode
\sphinxstyleliteralstrong{ax} \textendash{} matplotlib axes object or list of those

\end{description}\end{quote}

\end{fulllineitems}

\index{apply\_to\_fig() (config.Style method)}

\begin{fulllineitems}
\phantomsection\label{\detokenize{config:config.Style.apply_to_fig}}\pysiglinewithargsret{\sphinxbfcode{apply\_to\_fig}}{\emph{self}, \emph{fig}}{}~
\begin{DUlineblock}{0em}
\item[] Apply this style to the given figure and its axes (and new axes)  
\item[] You can apply to multiple figures by calling this multiple times or with a list of figures  
\end{DUlineblock}
\begin{quote}\begin{description}
\item[{Parameters}] \leavevmode
\sphinxstyleliteralstrong{fig} \textendash{} matplotlib figure or list of those

\end{description}\end{quote}

\end{fulllineitems}

\index{update() (config.Style method)}

\begin{fulllineitems}
\phantomsection\label{\detokenize{config:config.Style.update}}\pysiglinewithargsret{\sphinxbfcode{update}}{\emph{self}, \emph{E=None}, \emph{**F}}{}
Update the value in the internal dictionary and update the axes object
\begin{quote}\begin{description}
\item[{Parameters}] \leavevmode
\sphinxstyleliteralstrong{E} \textendash{} dictionary with key/value pairs to update or name of a predefined style

\end{description}\end{quote}

\end{fulllineitems}

\index{update\_style() (config.Style method)}

\begin{fulllineitems}
\phantomsection\label{\detokenize{config:config.Style.update_style}}\pysiglinewithargsret{\sphinxbfcode{update\_style}}{\emph{self}, \emph{E}}{}
more explicit alias to \sphinxcode{update}

\end{fulllineitems}

\index{reapply() (config.Style method)}

\begin{fulllineitems}
\phantomsection\label{\detokenize{config:config.Style.reapply}}\pysiglinewithargsret{\sphinxbfcode{reapply}}{\emph{self}}{}
reapply this style to the axes object

\end{fulllineitems}

\index{selectors() (config.Style method)}

\begin{fulllineitems}
\phantomsection\label{\detokenize{config:config.Style.selectors}}\pysiglinewithargsret{\sphinxbfcode{selectors}}{\emph{self}, \emph{print\_out=True}}{}
Print or Get a list of possible selectors

\end{fulllineitems}


\end{fulllineitems}

\phantomsection\label{\detokenize{animation:module-animation}}\index{animation (module)}

\chapter{Create animated plots}
\label{\detokenize{animation:animation}}\label{\detokenize{animation::doc}}\label{\detokenize{animation:create-animated-plots}}\index{create\_animation() (in module animation)}

\begin{fulllineitems}
\phantomsection\label{\detokenize{animation:animation.create_animation}}\pysiglinewithargsret{\sphinxcode{animation.}\sphinxbfcode{create\_animation}}{\emph{figure}, \emph{frame\_generator}, \emph{frames}, \emph{fps=None}, \emph{bitrate=18000}, \emph{dpi=100}, \emph{path=Noneblit=False}, \emph{clear\_between=False}, \emph{save\_args=None}, \emph{progress=True}, \emph{anim\_writer="ffmpeg"}}{}
Create an animation.
\begin{quote}\begin{description}
\item[{Parameters}] \leavevmode\begin{itemize}
\item {} 
\sphinxstyleliteralstrong{figure} \textendash{} The figure to use.

\item {} 
\sphinxstyleliteralstrong{frame\_generator} \textendash{} Callable that generates a new frame on figure!

\item {} 
\sphinxstyleliteralstrong{frames} \textendash{} List with index for each frame

\item {} 
\sphinxstyleliteralstrong{fps} \textendash{} The frames per seconds.

\item {} 
\sphinxstyleliteralstrong{bitrate} \textendash{} Bitrate of video

\item {} 
\sphinxstyleliteralstrong{dpi} \textendash{} Dpi of video

\item {} 
\sphinxstyleliteralstrong{path} \textendash{} Path to save the video to. If None does not save.

\item {} 
\sphinxstyleliteralstrong{save\_args} \textendash{} Keyword arguments for savefig

\item {} 
\sphinxstyleliteralstrong{progress} \textendash{} Show progress bar. If False do not show, if True show one, if integer use as  
offset for this bar

\end{itemize}

\end{description}\end{quote}

\end{fulllineitems}

\index{data\_frame\_generator() (in module animation)}

\begin{fulllineitems}
\phantomsection\label{\detokenize{animation:animation.data_frame_generator}}\pysiglinewithargsret{\sphinxcode{animation.}\sphinxbfcode{data\_frame\_generator}}{\emph{fig}, \emph{xdata}, \emph{ydata}, \emph{label\_only\_once=False}}{}
Generate a frame generator for simple data.
\begin{quote}\begin{description}
\item[{Parameters}] \leavevmode\begin{itemize}
\item {} 
\sphinxstyleliteralstrong{fig} \textendash{} The figure to draw in

\item {} 
\sphinxstyleliteralstrong{xdata} \textendash{} The xdata as iterable

\item {} 
\sphinxstyleliteralstrong{ydata} \textendash{} The ydata as iterable

\item {} 
\sphinxstyleliteralstrong{label\_only\_once} \textendash{} True to only show x and y tick labels only once  
(otherwise will be drawn over each other if axis present (not so nice))

\end{itemize}

\end{description}\end{quote}

\end{fulllineitems}

\phantomsection\label{\detokenize{utils:module-utils}}\index{utils (module)}

\chapter{Internal Utilities}
\label{\detokenize{utils:utils}}\label{\detokenize{utils:internal-utilities}}\label{\detokenize{utils::doc}}\begin{quote}\begin{description}
\item[{Author}] \leavevmode
Patrick Schreiber

\end{description}\end{quote}


\section{How the version handling for attributes works:}
\label{\detokenize{utils:how-the-version-handling-for-attributes-works}}
\sphinxtitleref{InovesaVersion} is a object that implements comparators for inovesa versions

\sphinxtitleref{unit\_to\_attr\_map} is a dictionary that maps physical units to inovesa attributes.  
Since these attributes change over different inovesa versions first a dictionary for “older”  
versions (13 and up, maybe some versions below match too) then it is modified into  
\sphinxtitleref{unit\_to\_attr\_map\_v15} for versions 15 and up (until the next change in inovesa is made).

\sphinxtitleref{attr\_to\_unit\_map} and \sphinxtitleref{attr\_to\_unit\_map\_v15} are just reversed dictionaries to make it possible  
to find the corresponding readable/printable unit for an attribute

Then \sphinxtitleref{alias\_map} is generated that makes it possible to not only use the exact specification  
used in unit\_to\_attr\_map but also the ones specified in \sphinxtitleref{alias\_map}. The reason for this is  
that the user does not have to remember the exact unit string.

Then a \sphinxtitleref{complete\_unit\_map} (and \sphinxtitleref{complete\_unit\_map\_v15}) is generated where all the entries in  
alias\_map are mapped to attributes

There are 3 ‘public’ methods that should be used.
\begin{itemize}
\item {} 
\sphinxtitleref{attr\_from\_unit}: Get the h5 attribute from the given unit

\item {} 
\sphinxtitleref{unit\_from\_attr}: Get the readable/printable unit from a h5 attribute

\item {} 
\sphinxtitleref{unit\_from\_spec}: Get the readable/printable unit from a unit specifier (an alias)

\end{itemize}
\index{UnitError (class in utils)}

\begin{fulllineitems}
\phantomsection\label{\detokenize{utils:utils.UnitError}}\pysiglinewithargsret{\sphinxbfcode{class }\sphinxcode{utils.}\sphinxbfcode{UnitError}}{\emph{Exception}}{}
\end{fulllineitems}

\index{DataNotInFile (class in utils)}

\begin{fulllineitems}
\phantomsection\label{\detokenize{utils:utils.DataNotInFile}}\pysiglinewithargsret{\sphinxbfcode{class }\sphinxcode{utils.}\sphinxbfcode{DataNotInFile}}{\emph{Exception}}{}
\end{fulllineitems}

\index{InovesaVersion (class in utils)}

\begin{fulllineitems}
\phantomsection\label{\detokenize{utils:utils.InovesaVersion}}\pysiglinewithargsret{\sphinxbfcode{class }\sphinxcode{utils.}\sphinxbfcode{InovesaVersion}}{\emph{object}}{}
Object to make comparison of versions easier

\end{fulllineitems}

\index{attr\_from\_unit() (in module utils)}

\begin{fulllineitems}
\phantomsection\label{\detokenize{utils:utils.attr_from_unit}}\pysiglinewithargsret{\sphinxcode{utils.}\sphinxbfcode{attr\_from\_unit}}{\emph{unit}, \emph{version}}{}
Get the h5 attribute from a unit specification
\begin{quote}\begin{description}
\item[{Parameters}] \leavevmode\begin{itemize}
\item {} 
\sphinxstyleliteralstrong{unit} \textendash{} specification to get the attribute for

\item {} 
\sphinxstyleliteralstrong{version} \textendash{} inovesa version as InovesaVersion Object

\end{itemize}

\item[{Returns}] \leavevmode
a string with the h5 attribute for the given unit

\item[{Raises}] \leavevmode
{\hyperref[\detokenize{utils:utils.UnitError}]{\sphinxcrossref{\sphinxstyleliteralstrong{UnitError}}}} \textendash{} when a unit is not known

\end{description}\end{quote}

\end{fulllineitems}

\index{unit\_from\_attr() (in module utils)}

\begin{fulllineitems}
\phantomsection\label{\detokenize{utils:utils.unit_from_attr}}\pysiglinewithargsret{\sphinxcode{utils.}\sphinxbfcode{unit\_from\_attr}}{\emph{attr}, \emph{version}}{}
Get a printable/readable unit from an h5 attribute
\begin{quote}\begin{description}
\item[{Parameters}] \leavevmode\begin{itemize}
\item {} 
\sphinxstyleliteralstrong{attr} \textendash{} h5 attribute to get the unit for

\item {} 
\sphinxstyleliteralstrong{version} \textendash{} inovesa version as InovesaVersion Object

\end{itemize}

\item[{Returns}] \leavevmode
a string with the requested unit

\item[{Raises}] \leavevmode
{\hyperref[\detokenize{utils:utils.UnitError}]{\sphinxcrossref{\sphinxstyleliteralstrong{UnitError}}}} \textendash{} when an attribute is not known

\end{description}\end{quote}

\end{fulllineitems}

\index{unit\_from\_spec() (in module utils)}

\begin{fulllineitems}
\phantomsection\label{\detokenize{utils:utils.unit_from_spec}}\pysiglinewithargsret{\sphinxcode{utils.}\sphinxbfcode{unit\_from\_spec}}{\emph{spec}}{}
Get a printable/readable unit from a unit specification
\begin{quote}\begin{description}
\item[{Parameters}] \leavevmode
\sphinxstyleliteralstrong{spec} \textendash{} the unit specification

\item[{Returns}] \leavevmode
a string with the requested unit

\item[{Raises}] \leavevmode
{\hyperref[\detokenize{utils:utils.UnitError}]{\sphinxcrossref{\sphinxstyleliteralstrong{UnitError}}}} \textendash{} when the specification is not known

\end{description}\end{quote}

\end{fulllineitems}

\index{calc\_stat\_mom() (in module utils)}

\begin{fulllineitems}
\phantomsection\label{\detokenize{utils:utils.calc_stat_mom}}\pysiglinewithargsret{\sphinxcode{utils.}\sphinxbfcode{calc\_stat\_mom}}{\emph{axis}, \emph{profiles}}{}
\end{fulllineitems}

\index{calc\_bl() (in module utils)}

\begin{fulllineitems}
\phantomsection\label{\detokenize{utils:utils.calc_bl}}\pysiglinewithargsret{\sphinxcode{utils.}\sphinxbfcode{calc\_bl}}{\emph{axis}, \emph{profiles}}{}
\end{fulllineitems}

\index{color\_map\_inferno() (in module utils)}

\begin{fulllineitems}
\phantomsection\label{\detokenize{utils:utils.color_map_inferno}}\pysiglinewithargsret{\sphinxcode{utils.}\sphinxbfcode{color\_map\_inferno}}{}{}
\end{fulllineitems}



\renewcommand{\indexname}{Python Module Index}
\begin{sphinxtheindex}
\def\bigletter#1{{\Large\sffamily#1}\nopagebreak\vspace{1mm}}
\bigletter{a}
\item {\sphinxstyleindexentry{animation}}\sphinxstyleindexpageref{animation:\detokenize{module-animation}}
\indexspace
\bigletter{c}
\item {\sphinxstyleindexentry{config}}\sphinxstyleindexpageref{config:\detokenize{module-config}}
\indexspace
\bigletter{d}
\item {\sphinxstyleindexentry{data}}\sphinxstyleindexpageref{data:\detokenize{module-data}}
\indexspace
\bigletter{f}
\item {\sphinxstyleindexentry{file}}\sphinxstyleindexpageref{file:\detokenize{module-file}}
\indexspace
\bigletter{p}
\item {\sphinxstyleindexentry{plots}}\sphinxstyleindexpageref{plots:\detokenize{module-plots}}
\indexspace
\bigletter{u}
\item {\sphinxstyleindexentry{utils}}\sphinxstyleindexpageref{utils:\detokenize{module-utils}}
\end{sphinxtheindex}

\renewcommand{\indexname}{Index}
\printindex
\end{document}